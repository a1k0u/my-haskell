\documentclass[a4paper,12pt]{article}
\usepackage{cmap}					% поиск в PDF
\usepackage{mathtext} 				% русские буквы в формулах
\usepackage[T2A]{fontenc}			% кодировка
\usepackage[utf8]{inputenc}			% кодировка исходного текста
\usepackage[english,russian]{babel}	% локализация и переносы
\usepackage{indentfirst}
\frenchspacing
\usepackage{amsmath,amsfonts,amssymb,amsthm,mathtools} % AMS
\usepackage{icomma} % "Умная" запятая: $0,2$ --- число, $0, 2$ --- перечисление
\DeclareMathOperator{\sgn}{\mathop{sgn}}
\newcommand*{\hm}[1]{#1\nobreak\discretionary{}
{\hbox{$\mathsurround=0pt #1$}}{}}
\usepackage{graphicx}  % Для вставки рисунков
\graphicspath{{images/}{images2/}}  % папки с картинками
\setlength\fboxsep{3pt} % Отступ рамки \fbox{} от рисунка
\setlength\fboxrule{1pt} % Толщина линий рамки \fbox{}
\usepackage{wrapfig} % Обтекание рисунков текстом
\usepackage{array,tabularx,tabulary,booktabs} % Дополнительная работа с таблицами
\usepackage{longtable}  % Длинные таблицы
\usepackage{multirow} % Слияние строк в таблице
\theoremstyle{plain} % Это стиль по умолчанию, его можно не переопределять.
\newtheorem{theorem}{Теорема}[section]
\newtheorem{proposition}[theorem]{Утверждение}
\theoremstyle{definition} % "Определение"
\newtheorem{corollary}{Следствие}[theorem]
\newtheorem{problem}{Задача}[section]
\theoremstyle{remark} % "Примечание"
\newtheorem*{nonum}{Решение}
\usepackage{etoolbox} % логические операторы
\usepackage{extsizes} % Возможность сделать 14-й шрифт
\usepackage{geometry} % Простой способ задавать поля
	\geometry{top=25mm}
	\geometry{bottom=35mm}
	\geometry{left=35mm}
	\geometry{right=20mm}
\usepackage{setspace} % Интерлиньяж
\usepackage{lastpage} % Узнать, сколько всего страниц в документе.
\usepackage{soul} % Модификаторы начертания
\usepackage{hyperref}
\usepackage[usenames,dvipsnames,svgnames,table,rgb]{xcolor}
\hypersetup{unicode=true, pdftitle={Заголовок}, pdfauthor={Автор},pdfsubject={Тема}, pdfcreator={Создатель}, pdfproducer={Производитель}, pdfkeywords={keyword1} {key2} {key3}, colorlinks=true, linkcolor=red, citecolor=black, filecolor=magenta, urlcolor=cyan}
\usepackage{csquotes} % Еще инструменты для ссылок
\usepackage{multicol} % Несколько колонок
\usepackage{tikz} % Работа с графикой
\usepackage{pgfplots}
\usepackage{pgfplotstable}

\date{ВШЭ ПМИ, 2022 г.} 
\author{Алексей Косенко}

\begin{document}

\maketitle

\section{Типизированное лямбда-исчисление.}
\subsection{Заселяем типы.}
\begin{itemize}
    \item $\lambda xyz.x \ : \ a \rightarrow b \rightarrow c \rightarrow a$
    \item $\lambda xyz.y \ : \ a \rightarrow b \rightarrow c \rightarrow b$
    \item $\lambda xyz.z \ : \ a \rightarrow b \rightarrow a \rightarrow a$
    \item $\lambda xyz.x \ : \ a \rightarrow a \rightarrow a \rightarrow a$
\end{itemize}

\subsection{Найдем обитателей типов.}
$\lambda xyzw. \ y \ (x \ w \ w) \ (z \ w) \ : \ (d \rightarrow d \rightarrow a) \rightarrow (a \rightarrow b \rightarrow c) \rightarrow (d \rightarrow b) \rightarrow d \rightarrow c$
\\

Для типа $(d \rightarrow d \rightarrow a) \rightarrow (c \rightarrow a) \rightarrow (a \rightarrow b) \rightarrow d \rightarrow c \rightarrow d$ существует два терма. Тело лямбда-функции должно быть типа $b$, есть только одна вершина, в которой получается этот тип, и два ребра, чтобы добраться до него.
\begin{itemize}
    \item $\lambda xyzwk. \ z \ (x \ w \ w)$
    \item $\lambda xyzwk. \ z \ (y \ k)$ 
\end{itemize}

\subsection{Типизируем по Чёрчу.}
Типизируем аппликацию комбинаторов по Чёрчу. 
$$S K K = (\lambda f^{\alpha \rightarrow (\beta \rightarrow \alpha) \rightarrow \alpha} g^{\alpha \rightarrow \beta \rightarrow \alpha} x^{\alpha}. \ f \ x \ (g \ x)) \ (\lambda x^{\alpha} y^{\beta \rightarrow \alpha}. x) \ (\lambda x^{\alpha} y^{\beta}. x) : \alpha \rightarrow \alpha$$
$$SKI = (\lambda f^{\alpha \rightarrow \alpha \rightarrow \alpha} g^{\alpha \rightarrow \alpha} x^{\alpha}. \ f \ x \ (g \ x)) \ (\lambda x^{\alpha} y^{\alpha}. \ x) \ (\lambda x^{\alpha}. \ x) \ : \ \alpha \rightarrow \alpha$$
\subsection{Конструируем термы с условием.}

Сконструируем терм типа: $(c \rightarrow e) \rightarrow ((c \rightarrow e) \rightarrow e) \rightarrow e$, которому нельзя будет приписать $a \rightarrow (a \rightarrow e) \rightarrow e$. Добьемся этого мы тем, что внутри лямбда-абстрации пропишем вложенную абстракцию. 

Итак, наша внутренняя лямбда-абстракция будет принимать тип $c$, а возвращать тип $e$, результатом последнего является аппликация внешних термов.

$$\lambda x y. \ y \ (\lambda z. \ y \ x) : (c \rightarrow e) \rightarrow ((c \rightarrow e) \rightarrow e) \rightarrow e$$


\subsection{Сложная конструкция.}
Для первого сформируем абстракцию, которая принимает $a$ и возвращает $b$ (\textit{аппликация в теле}), то есть $a \rightarrow b$, тогда получим:
$$\lambda x y. \ x \ (\lambda z. \ y \ z \ z) : ((a \rightarrow b) \rightarrow a) \rightarrow (a \rightarrow a \rightarrow b) \rightarrow a$$

Во втором будет дважды повторяться терм, который возврает $a \rightarrow b$ для того, чтобы путём аппликации с $x$ получить $a$.
$$\lambda xy. \ y \ (x \ (\lambda z. \ y \ z \ z)) \ (x \ (\lambda z. \ y \ z \ z)) : ((a \rightarrow b) \rightarrow a) \rightarrow (a \rightarrow a \rightarrow b) \rightarrow b$$

Третье, оно же самое вложенное, должны получить: $((((a \rightarrow b) \rightarrow a) \rightarrow a) \rightarrow b) \rightarrow b$. В самом вложенном получим $a \rightarrow b$, затем $((a \rightarrow b) \rightarrow a) \rightarrow a$, чтобы после получить $b$, апплицируя с $x$.
$$\lambda x. \ x \ (\lambda z_1^{(a \rightarrow b) \rightarrow a}. \ z_1 \ (\lambda z_2^{a}. \ (\lambda z_3^{a \rightarrow b}. \ z_3 \ z_2)))$$



\end{document}
