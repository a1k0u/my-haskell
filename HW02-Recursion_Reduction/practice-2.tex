\documentclass[a4paper,12pt]{article}
\usepackage{cmap}					% поиск в PDF
\usepackage{mathtext} 				% русские буквы в формулах
\usepackage[T2A]{fontenc}			% кодировка
\usepackage[utf8]{inputenc}			% кодировка исходного текста
\usepackage[english,russian]{babel}	% локализация и переносы
\usepackage{indentfirst}
\frenchspacing
\usepackage[linguistics]{forest}
\usepackage[latin1]{inputenc}
\usetikzlibrary{trees}
\usepackage{amsmath,amsfonts,amssymb,amsthm,mathtools} % AMS
\usepackage{icomma} % "Умная" запятая: $0,2$ --- число, $0, 2$ --- перечисление
\DeclareMathOperator{\sgn}{\mathop{sgn}}
\newcommand*{\hm}[1]{#1\nobreak\discretionary{}
{\hbox{$\mathsurround=0pt #1$}}{}}
\usepackage{graphicx}  % Для вставки рисунков
\graphicspath{{images/}{images2/}}  % папки с картинками
\setlength\fboxsep{3pt} % Отступ рамки \fbox{} от рисунка
\setlength\fboxrule{1pt} % Толщина линий рамки \fbox{}
\usepackage{wrapfig} % Обтекание рисунков текстом
\usepackage{array,tabularx,tabulary,booktabs} % Дополнительная работа с таблицами
\usepackage{longtable}  % Длинные таблицы
\usepackage{multirow} % Слияние строк в таблице
\theoremstyle{plain} % Это стиль по умолчанию, его можно не переопределять.
\newtheorem{theorem}{Теорема}[section]
\newtheorem{proposition}[theorem]{Утверждение} 
\theoremstyle{definition} % "Определение"
\newtheorem{corollary}{Следствие}[theorem]
\newtheorem{problem}{Задача}[section]
\theoremstyle{remark} % "Примечание"
\newtheorem*{nonum}{Решение}
\usepackage{etoolbox} % логические операторы
\usepackage{extsizes} % Возможность сделать 14-й шрифт
\usepackage{geometry} % Простой способ задавать поля
	\geometry{top=25mm}
	\geometry{bottom=35mm}
	\geometry{left=35mm}
	\geometry{right=20mm}
\usepackage{setspace} % Интерлиньяж
\usepackage{lastpage} % Узнать, сколько всего страниц в документе.
\usepackage{soul} % Модификаторы начертания
\usepackage{hyperref}
\usepackage[usenames,dvipsnames,svgnames,table,rgb]{xcolor}
\hypersetup{				% Гиперссылки
    unicode=true,           % русские буквы в раздела PDF
    pdftitle={Заголовок},   % Заголовок
    pdfauthor={Автор},      % Автор
    pdfsubject={Тема},      % Тема
    pdfcreator={Создатель}, % Создатель
    pdfproducer={Производитель}, % Производитель
    pdfkeywords={keyword1} {key2} {key3}, % Ключевые слова
    colorlinks=true,       	% false: ссылки в рамках; true: цветные ссылки
    linkcolor=red,          % внутренние ссылки
    citecolor=black,        % на библиографию
    filecolor=magenta,      % на файлы
    urlcolor=cyan           % на URL
}
\usepackage{csquotes} % Еще инструменты для ссылок
\usepackage{multicol} % Несколько колонок
\usepackage{tikz} % Работа с графикой
\usepackage{pgfplots}
\usepackage{pgfplotstable}

\author{Алексей Косенко}
\date{ВШЭ ПМИ, 2022 г.}

\begin{document}

\maketitle

\section{Рекурсия и редукция, практика 2.}

\subsection{Формы термов.}

$\lambda x. \ fst \ (pair \ x \ 0)$ -- терм находится в WHNF, но не в HNF.

$\lambda xy. \ x \ (\lambda z. \ y \ z) \ w$ -- находится в HNF, но не в NF.
\subsection{Функции для чисел Чёрча.}
\subsubsection{Equals.}
Заметим, что функция $minus$ с аргументами $n$, $m$, где второе число будет больше первого, будет всегда возвращать $0$.
$$minus \ 1 \ 2 = (\lambda n m. m \ pred \ n) \ 1 \ 2 \rightarrow^{\beta}$$
$$2 \ pred \ 1 = (\lambda sz. s \ (s \ z)) \ pred \ 1 \rightarrow^{\beta}$$
$$pred \ (pred \ 1) \twoheadrightarrow^{\beta} pred \ 0$$
$$pred \ 0 = (\lambda n. \ fst \ (n \ sp \ zp)) \ (\lambda sz. \ n) \rightarrow^{\beta}$$
$$fst \ ((\lambda sz. \ n) \ sp \ zp) \twoheadrightarrow^{\beta} fst \ (pair \ 0 \ 0) = 0$$
Для проверки равенства чисел достаточно вычесть их друг из друга, проверив результат на равенство нулю.
$$equals = \lambda n m. \ if \ (and \ (isZero \ (minus \ n \ m)) \ (isZero \ (minus \ m \ n))) \ tru \ fls$$

\subsubsection{lt, gt, le, ge.}
Для быстрого выведения первых двух формул используем $isZero$ и мини-трюк с $minus$. Затем на основе $lt, gt, equals$ выводим оставшиеся. 
\begin{align*}
lt &= \lambda m n. \ if \ (not \ (isZero \ (minus \ n \ m))) \ tru \ fls \\
gt &= \lambda m n. \ if \ (not \ (isZero \ (minus \ m \ n))) \ tru \ fls \\
le &= \lambda m n. \ if \ (or \ (lt \ m \ n) \ (equals \ m \ n)) \ tru \ fls \\
ge &= \lambda m n. \ if \ (or \ (gt \ m \ n) \ (equals \ m \ n)) \ tru \ fls
\end{align*}
\subsubsection{Sum.}
Просуммируем числа $[0, n)$. Воспользуемся идеей $pred$, но первым элементом пары будем хранить сумму предыдущих чисел, а на втором месте поддерживать счётчик. 

Расширим абстракцию $sp$ до $zsp = \lambda p. \ pair \ (plus \ (fst \ p) \ (snd \ p)) \ (succ \ (snd \ p))$. 
$$sum = \lambda n. \ fst \ (n \ zsp \ zp) $$
$$sum \ 0 = (\lambda n. \ fst \ (n \ zsp \ zp)) \ 0 \rightarrow^{\beta}$$
$$fst \ (0 \ zsp \ zp) \rightarrow^{\beta} fst \ zp = 0$$

$$sum \ 4 = (\lambda n. \ fst \ (n \ zsp \ zp)) \ 4 \rightarrow^{\beta}$$
$$fst \ (4 \ zsp \ zp) = fst \ ((\lambda sz. \ s \ (s \ (s \ (s \ z)))) \ zsp \ zp) \rightarrow^{\beta}$$
$$fst \ (zsp \ (zsp \ (zsp \ (zsp \ zp)))) \rightarrow^{\beta}$$
$$fst \ (zsp \ (zsp \ (zsp \ (pair \ 0 \ 1)))) \rightarrow^{\beta}$$
$$fst \ (zsp \ (zsp \ (pair \ 1 \ 2))) \rightarrow^{\beta}$$
$$fst \ (zsp \ (pair \ 3 \ 3)) \rightarrow^{\beta}$$
$$fst \ (pair \ 6 \ 4) = 6$$

\subsection{Остаток от деления.}
$$modCounter = \lambda p. \ if \ (equals \ p \ 3) \ 0 \ (succ \ p)$$

Введем лямбда-абстракцию $modCounter$, с помощью которой будем сбрасывать счетчик в $0$ при достижении трёх. При большом желании можно расширить $modCounter$ для любого числа, заменив захардкоженную константу в переменную и добавив её в абстрактор.
$$mod3 = \lambda m. \ if  \ (isZero \ m) \ 0 \ (m \ modCounter \ 1)$$

$$mod3 \ 3 \rightarrow^{\beta} \ if \ (isZero \ 3) \ 0 \ ((\lambda sz. \ s \ (s \ (s \ z))) \ modCounter \ 1) \rightarrow^{\beta}$$
$$modCounter \ (modCounter \ (modCounter \ 1)) \rightarrow^{\beta}$$
$$modCounter \ (modCounter \ 2) \rightarrow^{\beta}$$
$$modCounter \ 3 = 0 $$

$$mod3 \ 5 \twoheadrightarrow^{\beta} modCounter \ (modCounter \ (modCounter \ (modCounter \ (modCounter \ 1)))) \rightarrow^{\beta}$$
$$modCounter \ (modCounter \ (modCounter \ (modCounter \ 2))) \rightarrow^{\beta}$$
$$modCounter \ (modCounter \ (modCounter \ 3)) \rightarrow^{\beta}$$
$$modCounter \ (modCounter \ 0) \rightarrow^{\beta}$$
$$modCounter \ 1 \rightarrow^{\beta} 2$$


\subsection{Функции над списками.}

\subsubsection{Length.}
Создаем счётчик по второму элементу, инициализируя его с нуля.
$$counter = \lambda h k. \ succ \ k$$
$$length = \lambda l. \ l \ counter \ 0$$

$$length \ [1, 2, 3, 4] \rightarrow^{\beta} (\lambda cn. c \ 1 \ (c \ 2 \ (c \ 3 \ (c \ 4 \ n)))) \ counter \ 0 \rightarrow^{\beta}$$
$$counter \ 1 \ (counter \ 2 \ (counter \ 3 \ (counter \ 4 \ 0))) \rightarrow^{\beta}$$
$$counter \ 1 \ (counter \ 2 \ (counter \ 3 \ 1)) \rightarrow^{\beta}$$
$$counter \ 1 \ (counter \ 2 \ 2) \rightarrow^{\beta}$$
$$counter \ 1 \ 3 = 4$$

$$length \ [ \ ] \rightarrow^{\beta} (\lambda cn. \ n) \ counter \ 0 = 0$$

\subsubsection{Tail.}
Создаем счётчик от $length - 1$, он будет конкатенировать до тех пор, пока не достигнет нуля. 

Идея также схожа с реализацией $pred$, только вместо $c$ подставляем $adder$. Время работы $\Theta (n)$.

$Adder$ на каждой итерации добавляет в массив новое значение и уменьшает счётчик на единицу. В конце будет возвращен массив без первого элемента.
$$adder = \lambda e p. \ if \ (isZero \ (snd \ p)) \ (fst \ p) \  (pair \ (cons \ e \ (fst \ p)) \ (pred \ (snd \ p)))$$
$$tail = \lambda l. \ if \ (isEmpty \ l) \ [ \ ] \ ( l \ adder \ (pair \ nil \ (pred \  (length \ l)))) $$

$$tail \ [5, 3, 4, 1] \twoheadrightarrow^{\beta} (\lambda c n. \ c \ 5 \ (c \ 3 \ (c \ 4 \ (c \ 1 \ n)))) \  adder \ (pair \ [ \ ] \ 3)$$
$$ adder \ 5 \ (adder \ 3 \ (adder \ 4 \ (adder \ 1 \ (pair \ [ \ ] \ 3)))) \rightarrow^{\beta}$$
$$ adder \ 5 \ (adder \ 3 \ (adder \ 4 \ (pair \ [ 1 ] \ 2))) \rightarrow^{\beta}$$
$$ adder \ 5 \ (adder \ 3 \ (pair \ [ 4, 1 ] \ 1)) \rightarrow^{\beta}$$
$$ adder \ 5 \ (pair \ [ 3, 4, 1 ] \ 0) \rightarrow^{\beta} [ 3, 4, 1 ]$$

$$tail \ [5] \twoheadrightarrow^{\beta} (\lambda c n. \ c \ 5 \ n) \  adder \ (pair \ [ \ ] \ 0) \rightarrow^{\beta}$$
$$adder \ 5 \ (pair \ [ \ ] \ 0) \rightarrow^{\beta} [ \ ]$$

\subsection{Уравнения над термами.}
Используя комбинатор неподвижной точки найдём терм F такой что:

\begin{align*} 
FM &= MF \\
FM &= (\lambda m. \ m \ F) \ M \\ 
F  &= (\lambda m. \ m \ F) \\ 
F  &= (\lambda fm. \ m \ f) \ F \\
F' &= (\lambda fm. \ m \ f) \\
F  &= YF' \\
F  &= Y(\lambda f m. \ m \ f) 
\end{align*}
\begin{align*} 
FMN &= NF(MNF) \\
FMN &= (\lambda n. \ n \ F \ (M \ n \ F)) \ N \\
FM  &= \lambda n. \ n \ F \ (M \ n \ F) \\
FM  &= (\lambda m n. \ n \ F \ (m \ n \ F)) \ M \\
F   &= \lambda m n. \ n \ F \ (m \ n \ F) \\
F   &= (\lambda f m n. \ n \ f \ (m \ n \ f)) \ F \\
F   &= Y(\lambda f m n. \ n \ f \ (m \ n \ f)) 
\end{align*}
\subsection{Нерекурсивные определения функций.}
Пусть f и g определены взаимно-рекурсивно. Используя комбинатор неподвижной точки найдем нерекурсивные определения функций f и g.

$$\bold{f = F \ f \ g}$$
$$\bold{g = G \ f \ g}$$

$$M = pair \ f \ g$$
$$M = pair \ (F \ (fst \ M) \ (snd \ M)) \ (G \ (fst \ M) \ (snd \ M))$$
$$M = (\lambda P. pair \ (F \ (fst \ P) \ (snd \ P)) \ (G \ (fst \ P) \ (snd \ P))) \ M$$
$$M = Y(\lambda P. pair \ (F \ (fst \ P) \ (snd \ P)) \ (G \ (fst \ P) \ (snd \ P)))$$
$$K = (\lambda P. pair \ (F \ (fst \ P) \ (snd \ P)) \ (G \ (fst \ P) \ (snd \ P)))$$
$$M = YK$$

$$f = fst \ YK$$
$$g = snd \ YK$$


\subsection{N-ное число Фибоначчи.}
$F_0 = 0, F_1 = 1, F_n = F_{n - 1} + F_{n - 2}$.
$$prev = \lambda p. \ pair \ (plus \ (fst \ p) \ (snd \ p)) \ (snd \ p)$$
$$fib = \lambda n. \ snd (n \ prev \ (pair \ 1 \ 0))$$

$$fib \ 5 \twoheadrightarrow^{\beta} snd \ (5 \ prev \ (pair \ 1 \ 0)) \rightarrow^{\beta}$$
$$snd \ (prev \ (prev \ (prev \ (prev \ (prev \  (pair \ 1 \ 0)))))) \rightarrow^{\beta}$$
$$snd \ (prev \ (prev \ (prev \ (prev \ (pair \ 1 \ 1))))) \rightarrow^{\beta}$$
$$snd \ (prev \ (prev \ (prev \ (pair \ 2 \ 1)))) \rightarrow^{\beta}$$
$$snd \ (prev \ (prev \ (pair \ 3 \ 2))) \rightarrow^{\beta}$$
$$snd \ (prev \ (pair \ 5 \ 3)) \rightarrow^{\beta}$$
$$snd \ (pair \ 8 \ 5) \rightarrow^{\beta} 5$$

$$fib \ 0 \rightarrow^{\beta} 0$$

\end{document}
