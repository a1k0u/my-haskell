\documentclass[a4paper,12pt]{article}

%%% Работа с русским языком
\usepackage{cmap}					% поиск в PDF
\usepackage{mathtext} 				% русские буквы в формулах
\usepackage[T2A]{fontenc}			% кодировка
\usepackage[utf8]{inputenc}			% кодировка исходного текста
\usepackage[english,russian]{babel}	% локализация и переносы
\usepackage{indentfirst}
\frenchspacing


%%% Дополнительная работа с математикой
\usepackage{amsmath,amsfonts,amssymb,amsthm,mathtools} % AMS
\usepackage{icomma} % "Умная" запятая: $0,2$ --- число, $0, 2$ --- перечисление

%% Номера формул
%\mathtoolsset{showonlyrefs=true} % Показывать номера только у тех формул, на которые есть \eqref{} в тексте.
%\usepackage{leqno} % Нумерация формул слева

%% Свои команды
\DeclareMathOperator{\sgn}{\mathop{sgn}}

%% Перенос знаков в формулах (по Львовскому)
\newcommand*{\hm}[1]{#1\nobreak\discretionary{}
{\hbox{$\mathsurround=0pt #1$}}{}}

%%% Работа с картинками
\usepackage{graphicx}  % Для вставки рисунков
\graphicspath{{images/}{images2/}}  % папки с картинками
\setlength\fboxsep{3pt} % Отступ рамки \fbox{} от рисунка
\setlength\fboxrule{1pt} % Толщина линий рамки \fbox{}
\usepackage{wrapfig} % Обтекание рисунков текстом

%%% Работа с таблицами
\usepackage{array,tabularx,tabulary,booktabs} % Дополнительная работа с таблицами
\usepackage{longtable}  % Длинные таблицы
\usepackage{multirow} % Слияние строк в таблице

%%% Теоремы
\theoremstyle{plain} % Это стиль по умолчанию, его можно не переопределять.
\newtheorem{theorem}{Теорема}[section]
\newtheorem{proposition}[theorem]{Утверждение}
 
\theoremstyle{definition} % "Определение"
\newtheorem{corollary}{Следствие}[theorem]
\newtheorem{problem}{Задача}[section]
 
\theoremstyle{remark} % "Примечание"
\newtheorem*{nonum}{Решение}

%%% Программирование
\usepackage{etoolbox} % логические операторы

%%% Страница
\usepackage{extsizes} % Возможность сделать 14-й шрифт
\usepackage{geometry} % Простой способ задавать поля
	\geometry{top=25mm}
	\geometry{bottom=35mm}
	\geometry{left=35mm}
	\geometry{right=20mm}
 %
%\usepackage{fancyhdr} % Колонтитулы
% 	\pagestyle{fancy}
 	%\renewcommand{\headrulewidth}{0pt}  % Толщина линейки, отчеркивающей верхний колонтитул
% 	\lfoot{Нижний левый}
% 	\rfoot{Нижний правый}
% 	\rhead{Верхний правый}
% 	\chead{Верхний в центре}
% 	\lhead{Верхний левый}
%	\cfoot{Нижний в центре} % По умолчанию здесь номер страницы

\usepackage{setspace} % Интерлиньяж
%\onehalfspacing % Интерлиньяж 1.5
%\doublespacing % Интерлиньяж 2
%\singlespacing % Интерлиньяж 1

\usepackage{lastpage} % Узнать, сколько всего страниц в документе.

\usepackage{soul} % Модификаторы начертания

\usepackage{hyperref}
\usepackage[usenames,dvipsnames,svgnames,table,rgb]{xcolor}
\hypersetup{				% Гиперссылки
    unicode=true,           % русские буквы в раздела PDF
    pdftitle={Заголовок},   % Заголовок
    pdfauthor={Автор},      % Автор
    pdfsubject={Тема},      % Тема
    pdfcreator={Создатель}, % Создатель
    pdfproducer={Производитель}, % Производитель
    pdfkeywords={keyword1} {key2} {key3}, % Ключевые слова
    colorlinks=true,       	% false: ссылки в рамках; true: цветные ссылки
    linkcolor=red,          % внутренние ссылки
    citecolor=black,        % на библиографию
    filecolor=magenta,      % на файлы
    urlcolor=cyan           % на URL
}

\usepackage{csquotes} % Еще инструменты для ссылок

%\usepackage[style=authoryear,maxcitenames=2,backend=biber,sorting=nty]{biblatex}

\usepackage{multicol} % Несколько колонок

\usepackage{tikz} % Работа с графикой
\usepackage{pgfplots}
\usepackage{pgfplotstable}

\begin{document}

\section{Чистое лямбда-исчисление, практика 1.}
\subsection{Покажем, что}

$$B \ \left(=^{?}\right) \ S(KS)K$$

$$S(KS)K = (\lambda fgx.f \ x \ (g \ x) ) \ (KS) \ K \rightarrow^{\beta}$$
$$(\lambda gx.(KS) \ x \ (g \ x)) \ K = (\lambda gx.((\lambda m n. m) \ S) \ x \ (g \ x)) \ K \rightarrow^{\beta}$$
$$(\lambda gx.(\lambda n. S) \ x \ (g \ x)) \ K \rightarrow^{\beta}$$
$$(\lambda gx. S \ (g \ x)) \ K = (\lambda gx.  (\lambda f  h y. f  \ y \ (h \ y)) \ (g \ x)) \ K  \rightarrow^{\beta} $$
$$(\lambda gx.  (\lambda h y. (g \ x)  \ y \ (h \ y))  \ K  = \dots  \ (\lambda m n. m) \rightarrow^{\beta}$$
$$(\lambda x.  (\lambda h y. ((\lambda m n. m) \ x)  \ y \ (h \ y)) \rightarrow^{\beta}$$
$$(\lambda x.  (\lambda h y. (\lambda n. x)  \ y \ (h \ y)) \rightarrow^{\beta}$$
$$(\lambda x.  (\lambda h y. x \ (h \ y)) = \lambda xhy.x \ (h \ y) \rightarrow^{\alpha}$$
$$\lambda fgx.f \ (g \ x)$$

$$S(KS)K \twoheadrightarrow B \Rightarrow S(KS)K =^{\beta} B \Rightarrow B =^{\beta} S(KS)K \ \ \ \ \ \blacksquare $$

$$\rule{25em}{0.01em}$$

$$K^{*} \ \left(=^{?}\right) \ KI$$
$$KI = (\lambda x y. x) \ I \rightarrow^{\beta}$$
$$(\lambda y. I) = (\lambda y. (\lambda x. x)) = \lambda y x. x \rightarrow^{\alpha}$$
$$\lambda x y. y = K^* \ \ \ \ \ \blacksquare$$

\subsection{Выделим свободные и связанные переменные в термах и осуществим подстановки.}

Свободные переменные в терме выделены: $\boldsymbol{x} \ (\lambda x y. y \ (x \ \boldsymbol{w}) \ \boldsymbol{u}) \ \boldsymbol{y}$, остальные - \textit{связанные}. Осуществим подстановку [$x := \lambda z. z$]:

$$ (\lambda z. z) \ (\lambda x y. y \ (x \ w) \ u) \ y $$

Свободные переменные в терме выделены: $(\lambda x. x \ (\lambda y. y \ x) \ \boldsymbol{w}) (\lambda x. \boldsymbol{v})$, ($FV = \{w, \ v\}$), остальные - \textit{связанные}. Осуществим подстановку [$w := y \ (\lambda v. v \ x)$]:

$$(\lambda x'. x' \ (\lambda y'. y' \ x') \ y \ (\lambda v. v \ x)) (\lambda x. v)$$

\subsection{Уберем лишние скобки и осуществим бета-преобразования.}
$((\lambda x. (\lambda y. ((x \ y) \ z))) \ (a \ (b \ c))) \longrightarrow (\lambda x y. x \ y \ z) \ (a \ (b \ c)) \rightarrow^{\beta} \lambda y. (a \ (b \ c)) \ y \ z$ 

$ $

$(((m \ n) \ m) \ ( \lambda x. ((x \ (u \ v)) \ y))) \longrightarrow m \ n \ m \ ( \lambda x. x \ (u \ v) \ y)$

\subsection{XOR как терм.}

$fls = \lambda t f. f$

$tru = \lambda t f. t$

$\boldsymbol{XOR}$ = $\lambda xy. x \ y \ (y \ tru \ fls) \ fls \ tru$

\subsection{Арифметические операции с числами Чёрча.}


\end{document}
