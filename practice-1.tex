\documentclass[a4paper,12pt]{article}

%%% Работа с русским языком
\usepackage{cmap}					% поиск в PDF
\usepackage{mathtext} 				% русские буквы в формулах
\usepackage[T2A]{fontenc}			% кодировка
\usepackage[utf8]{inputenc}			% кодировка исходного текста
\usepackage[english,russian]{babel}	% локализация и переносы
\usepackage{indentfirst}
\frenchspacing


%%% Дополнительная работа с математикой
\usepackage{amsmath,amsfonts,amssymb,amsthm,mathtools} % AMS
\usepackage{icomma} % "Умная" запятая: $0,2$ --- число, $0, 2$ --- перечисление

%% Номера формул
%\mathtoolsset{showonlyrefs=true} % Показывать номера только у тех формул, на которые есть \eqref{} в тексте.
%\usepackage{leqno} % Нумерация формул слева

%% Свои команды
\DeclareMathOperator{\sgn}{\mathop{sgn}}

%% Перенос знаков в формулах (по Львовскому)
\newcommand*{\hm}[1]{#1\nobreak\discretionary{}
{\hbox{$\mathsurround=0pt #1$}}{}}

%%% Работа с картинками
\usepackage{graphicx}  % Для вставки рисунков
\graphicspath{{images/}{images2/}}  % папки с картинками
\setlength\fboxsep{3pt} % Отступ рамки \fbox{} от рисунка
\setlength\fboxrule{1pt} % Толщина линий рамки \fbox{}
\usepackage{wrapfig} % Обтекание рисунков текстом

%%% Работа с таблицами
\usepackage{array,tabularx,tabulary,booktabs} % Дополнительная работа с таблицами
\usepackage{longtable}  % Длинные таблицы
\usepackage{multirow} % Слияние строк в таблице

%%% Теоремы
\theoremstyle{plain} % Это стиль по умолчанию, его можно не переопределять.
\newtheorem{theorem}{Теорема}[section]
\newtheorem{proposition}[theorem]{Утверждение}
 
\theoremstyle{definition} % "Определение"
\newtheorem{corollary}{Следствие}[theorem]
\newtheorem{problem}{Задача}[section]
 
\theoremstyle{remark} % "Примечание"
\newtheorem*{nonum}{Решение}

%%% Программирование
\usepackage{etoolbox} % логические операторы

%%% Страница
\usepackage{extsizes} % Возможность сделать 14-й шрифт
\usepackage{geometry} % Простой способ задавать поля
	\geometry{top=25mm}
	\geometry{bottom=35mm}
	\geometry{left=35mm}
	\geometry{right=20mm}
 %
%\usepackage{fancyhdr} % Колонтитулы
% 	\pagestyle{fancy}
 	%\renewcommand{\headrulewidth}{0pt}  % Толщина линейки, отчеркивающей верхний колонтитул
% 	\lfoot{Нижний левый}
% 	\rfoot{Нижний правый}
% 	\rhead{Верхний правый}
% 	\chead{Верхний в центре}
% 	\lhead{Верхний левый}
%	\cfoot{Нижний в центре} % По умолчанию здесь номер страницы

\usepackage{setspace} % Интерлиньяж
%\onehalfspacing % Интерлиньяж 1.5
%\doublespacing % Интерлиньяж 2
%\singlespacing % Интерлиньяж 1

\usepackage{lastpage} % Узнать, сколько всего страниц в документе.

\usepackage{soul} % Модификаторы начертания

\usepackage{hyperref}
\usepackage[usenames,dvipsnames,svgnames,table,rgb]{xcolor}
\hypersetup{				% Гиперссылки
    unicode=true,           % русские буквы в раздела PDF
    pdftitle={Заголовок},   % Заголовок
    pdfauthor={Автор},      % Автор
    pdfsubject={Тема},      % Тема
    pdfcreator={Создатель}, % Создатель
    pdfproducer={Производитель}, % Производитель
    pdfkeywords={keyword1} {key2} {key3}, % Ключевые слова
    colorlinks=true,       	% false: ссылки в рамках; true: цветные ссылки
    linkcolor=red,          % внутренние ссылки
    citecolor=black,        % на библиографию
    filecolor=magenta,      % на файлы
    urlcolor=cyan           % на URL
}

\usepackage{csquotes} % Еще инструменты для ссылок

%\usepackage[style=authoryear,maxcitenames=2,backend=biber,sorting=nty]{biblatex}

\usepackage{multicol} % Несколько колонок

\usepackage{tikz} % Работа с графикой
\usepackage{pgfplots}
\usepackage{pgfplotstable}

\date{ВШЭ ПМИ, 2022 г.} 
\author{Алексей Косенко}

\begin{document}

\maketitle
\section{Чистое лямбда-исчисление, практика 1.}
\subsection{Покажем, что}

$$B \ \left(=^{?}\right) \ S(KS)K$$

$$S(KS)K = (\lambda fgx.f \ x \ (g \ x) ) \ (KS) \ K \rightarrow^{\beta}$$
$$(\lambda gx.(KS) \ x \ (g \ x)) \ K = (\lambda gx.((\lambda m n. m) \ S) \ x \ (g \ x)) \ K \rightarrow^{\beta}$$
$$(\lambda gx.(\lambda n. S) \ x \ (g \ x)) \ K \rightarrow^{\beta}$$
$$(\lambda gx. S \ (g \ x)) \ K = (\lambda gx.  (\lambda f  h y. f  \ y \ (h \ y)) \ (g \ x)) \ K  \rightarrow^{\beta} $$
$$(\lambda gx.  (\lambda h y. (g \ x)  \ y \ (h \ y))  \ K  = \dots  \ (\lambda m n. m) \rightarrow^{\beta}$$
$$(\lambda x.  (\lambda h y. ((\lambda m n. m) \ x)  \ y \ (h \ y)) \rightarrow^{\beta}$$
$$(\lambda x.  (\lambda h y. (\lambda n. x)  \ y \ (h \ y)) \rightarrow^{\beta}$$
$$(\lambda x.  (\lambda h y. x \ (h \ y)) = \lambda xhy.x \ (h \ y) \rightarrow^{\alpha}$$
$$\lambda fgx.f \ (g \ x)$$

$$S(KS)K \twoheadrightarrow B \Rightarrow S(KS)K =^{\beta} B \Rightarrow B =^{\beta} S(KS)K \ \ \ \ \ \blacksquare $$

$$\rule{25em}{0.01em}$$

$$K^{*} \ \left(=^{?}\right) \ KI$$
$$KI = (\lambda x y. x) \ I \rightarrow^{\beta}$$
$$(\lambda y. I) = (\lambda y. (\lambda x. x)) = \lambda y x. x \rightarrow^{\alpha}$$
$$\lambda x y. y = K^* \ \ \ \ \ \blacksquare$$

\subsection{Выделим свободные и связанные переменные в термах и осуществим подстановки.}

Свободные переменные в терме выделены: $\boldsymbol{x} \ (\lambda x y. y \ (x \ \boldsymbol{w}) \ \boldsymbol{u}) \ \boldsymbol{y}$, остальные - \textit{связанные}. Осуществим подстановку [$x := \lambda z. z$]:

$$ (\lambda z. z) \ (\lambda x y. y \ (x \ w) \ u) \ y $$

Свободные переменные в терме выделены: $(\lambda x. x \ (\lambda y. y \ x) \ \boldsymbol{w}) (\lambda x. \boldsymbol{v})$, ($FV = \{w, \ v\}$), остальные - \textit{связанные}. Осуществим подстановку [$w := y \ (\lambda v. v \ x)$]:

$$(\lambda x'. x' \ (\lambda y'. y' \ x') \ y \ (\lambda v. v \ x)) (\lambda x. v)$$

\subsection{Уберем лишние скобки и осуществим бета-преобразования.}
$((\lambda x. (\lambda y. ((x \ y) \ z))) \ (a \ (b \ c))) \longrightarrow (\lambda x y. x \ y \ z) \ (a \ (b \ c)) \rightarrow^{\beta} \lambda y. a \ (b \ c) \ y \ z$ 

$ $

$(((m \ n) \ m) \ ( \lambda x. ((x \ (u \ v)) \ y))) \longrightarrow m \ n \ m \ ( \lambda x. x \ (u \ v) \ y)$

\subsection{XOR как терм.}

$fls = \lambda t f. f$

$tru = \lambda t f. t$

$\boldsymbol{XOR}$ = $\lambda xy. x \ y \ (y \ tru \ fls) \ fls \ tru$

\subsection{Арифметические операции с числами Чёрча.}

\subsubsection{Plus.}

С помощью математической индукции проверим работу $plus', mult'$, которые в дальнейшем будем записывать - "$plus, mult$". Заранее упомянем анонимную функцию $succ = \lambda nsz. s(n \ s \ z)$, с помощью которой будем выполнять шаг идукции, правдивость её работы была доказана на семинаре.

Итак, $plus = \lambda m n s z. m \ s \ (n \ s \ z)$. Пусть числа Чёрча для доказательства базы индукции будут равны: $n = 2$ и $m = 1$. 

$$plus \ 1 \ 2 = (\lambda m n sz. m \ s \ ( n \ s \ z)) \ 1 \ 2 =^{\beta}$$
$$(\lambda n sz. (\lambda s' z'. s'z') \ s \ ( n \ s \ z)) \ 2 =^{\beta}$$
$$(\lambda n sz. (\lambda z'. sz') \ ( n \ s \ z)) \ 2 =^{\beta}$$
$$(\lambda n sz. s ( n \ s \ z)) \ 2 =^{\beta}$$
$$\lambda sz. s ( (\lambda s'z'. s'(s'z')) \ s \ z)=^{\beta}$$
$$\lambda sz. s ( (\lambda z'. s(sz')) \ z)=^{\beta}$$
$$\lambda sz. s ( s \ (s \ z))= 3$$

Предположим, что при $p$ и $k$ выполняется  $plus \ p \ k$, тогда докажем, что $plus \ p \ (succ \ k)$ также выполняется, то есть результат будет равен $int(p) + int(k) + 1$. 

$$plus \ p \ (succ \ k) = (\lambda m n s z. m \ s \ (n \ s \ z)) \ p \ (succ \ k) =^{\beta}$$
$$(\lambda n s z. (\lambda s'z'. s'(s'(\dots(s'z')\dots))) \ s \ (n \ s \ z)) \ (succ \ k) =^{\beta}$$
$$(\lambda n s z. (\lambda z'. s(s(\dots(s \ z')\dots))) \ (n \ s \ z)) \ (succ \ k) =^{\beta}$$
$$(\lambda n s z. s(s(\dots(s \ (n \ s \ z))\dots))) \ (succ \ k) =^{\beta}$$
$$\lambda s z. s(s(\dots(s \ ((\lambda ns'z'. s'(n \ s' \ z')) \ k \ s \ z))\dots)) =^{\beta}$$
$$\lambda s z. s(s(\dots(s \ ((\lambda s'z'. s'((\lambda s''z''. s''(s''(\dots(s''z'')\dots))) \ s' \ z')) \ s \ z))\dots)) =^{\beta}$$
$$\lambda s z. s(s(\dots(s \ ((\lambda s'z'. s'(s'(s'(\dots(s'z')\dots)))) \ s \ z))\dots)) =^{\beta}$$
$$\lambda s z. s(s(\dots(s \ (s(s(s(\dots(sz)\dots)))))\dots)) = p + (k + 1)$$

Каждое число Чёрча можно охарактеризовать количеством $s$ в записи, то есть число $p$ имеет $int(p)$ повторений $s$ в теле лямбда-абстракции, анологично для $k$. Грубо говоря, когда мы применили $\beta$-редукцию к $p$, то в абстракторе у числа $p$ осталось $z'$, куда мы редуцируем наше второе число, увеличенное на 1. Значит, что было $p$ раз повторений $s$, а теперь мы получаем вместо $z'$ ещё $k+1$ повторений $s$, следовательно количество $s$ равно $p+k+1$, что являтся число Чёрча $p + k + 1$, поэтому результат индукционный шага доказан.    

\subsubsection{Mult.}

База индукции для $mult$: $m = 2$ и $n = 3$.

$$mult \ 2 \ 3 = (\lambda m n s z. m \ (n \ s) \ z) \ 2 \ 3 =^{\beta}$$
$$(\lambda n s z. (\lambda s'z'. s'(s'z')) \ (n \ s) \ z) \ 3 =^{\beta}$$
$$(\lambda n s z. (\lambda z'. (n \ s)((n \ s) \ z')) \ z) \ 3 =^{\beta}$$
$$(\lambda n s z. (n \ s)((n \ s) \ z)) \ 3 =^{\beta}$$
$$\lambda s z. ((\lambda s'z'.s'(s'(s'z'))) \ s)(((\lambda s'z'.s'(s'(s'z'))) \ s) \ z) =^{\beta}$$
$$\lambda s z. (\lambda z'.s(s(sz')))((\lambda z'.s(s(sz'))) \ z) =^{\beta}$$
$$\lambda s z. (\lambda z'.s(s(sz')))(s(s(sz))) =^{\beta}$$
$$\lambda s z. s(s(s(s(s(sz))))) = 6$$

Допустим, что для $p$ и $k$ выполняется $mult \ p \ k$, тогда докажем, что для $mult \ p \ (succ \ k)$ результат также верен.

$$mult \ p \ (succ \ k) = (\lambda m n s z. m \ (n \ s) \ z) \ p \ (succ \ k) =^{\beta}$$
$$(\lambda n s z. (\lambda s'z'.s'(s'(\dots(s'z')\dots))) \ (n \ s) \ z) \ (succ \ k) =^{\beta}$$
$$(\lambda n s z. (\lambda z'.(n \ s)((n \ s)(\dots((n \ s) \ z')\dots))) \ z) \ (succ \ k) =^{\beta}$$
$$(\lambda n s z. (n \ s)((n \ s)(\dots((n \ s) \ z)\dots))) \ (succ \ k) =^{\beta}$$
$$\lambda s z. ((succ \ k) \ s)(((succ \ k) \ s)(\dots((succ \ k) \ s \ z)\dots)) =$$
$$\lambda s z. ((succ \ k) \ s)(((succ \ k) \ s)(\dots((succ \ k) \ s \ ((succ \ k) \ s \ z))\dots)) =$$


Заметим, что $\lambda sz. (succ \ k) \ s \ ((succ \ k) \ s \ z)$ - это $plus \ (succ \ k) \ (succ \ k)$. В предыдущий раз мы доказали верность работы $plus$, поэтому сейчас мы имеем полное право воспользоваться этим. 
Итак, на данный момент $succ \ k$, который стоит равно $p$ раз вместо предыдущий $s$. Можем записать результат как сложение:

$$\lambda sz. plus \ (succ \ k) (plus \ (succ \  k) (\dots (plus \ (succ \ k) \ (succ \ k)) \dots))$$

Запись выше означает сложение $(k + 1)$ $p$-раз, что и есть $p \cdot (k + 1)$. Однако можно рассуждать иначе (чем-то схоже с $plus$). Каждый раз при раскрытие внутренних скобок мы получаем число, в котором на $k+1$ больше $s$, что характеризует число увеличенное на $k+1$. Делаем так, раскрывая все скобки, и получаем $p \cdot (k + 1)$.

$$\lambda s z. ((succ \ k) \ s)(\dots((succ \ k) \ s \ ((\lambda s'z'. s'((\lambda s''z''. s''(\dots (s''z'') \dots))) \ s' \ z') \ s \ z))\dots) =^{\beta}$$
$$\lambda s z. ((succ \ k) \ s)(\dots((succ \ k) \ s \ ((\lambda s'z'. s'(s'(\dots (s'z') \dots))) \ s \ z))\dots) =^{\beta}$$
$$\lambda s z. ((succ \ k) \ s)(\dots((succ \ k) \ s \ (s(s(\dots (sz) \dots))))\dots) =^{\beta}$$
$$\lambda s z. ((succ \ k) \ s)(\dots((\lambda s'z'. s'((\lambda s''z''. s''(\dots (s''z'') \dots))) \ s' \ z') \ s \ (s(s(\dots (sz) \dots))))\dots) =^{\beta}$$
$$\lambda s z. ((succ \ k) \ s)(\dots((\lambda s'z'. s'(s'(\dots (s'z') \dots))) \ s \ (s(s(\dots (sz) \dots))))\dots) =^{\beta}$$
$$\lambda s z. ((succ \ k) \ s)(\dots((\lambda z'. s(s(\dots (sz') \dots))) \ (s(s(\dots (sz) \dots))))\dots) =^{\beta}$$
$$\lambda s z. ((succ \ k) \ s)(\dots(s(s(\dots (s(s(s(\dots (sz) \dots))))\dots)) \dots))$$

Так мы и получаем $p\cdot (k + 1)$, проводя $\beta$-редукцию. $\ \ \ \ \ \blacksquare$

\subsubsection{Power.}

База индукции для $power = \lambda m n s z. \ n \ m \ s \ z$ при $n = 2$ и $m = 3$.

$$power \ 3 \ 2 = (\lambda m n s z. n \ m \ s \ z) \ 3 \ 2 \rightarrow^{\beta}$$
$$(\lambda n s z. n \ (\lambda s'z'. s'(s'(s'z'))) \ s \ z) \ 2 \rightarrow^{\beta}$$
$$\lambda s z. (\lambda s''z''. s''(s''z'')) \ 3 \ s \ z \rightarrow^{\beta}$$
$$\lambda s z. (\lambda z''. 3 \ (3 \ z'')) \ s \ z \rightarrow^{\beta}$$
$$\lambda s z. 3 \ (3 \ s) \ z =$$
$$\lambda s z. 3 \ ((\lambda s'z'. s'(s'(s'z'))) \ s) \ z \rightarrow^{\beta}$$
$$\lambda s z. (\lambda s''z''. s''(s''(s''z''))) \ (\lambda z'. s(s(sz'))) \ z \rightarrow^{\beta}$$
$$\lambda s z. (\lambda z''. (\lambda z'. s(s(sz')))((\lambda z'. s(s(sz')))((\lambda z'. s(s(sz')))z''))) \ z \rightarrow^{\beta}$$
$$\lambda s z. (\lambda z'. s(s(sz')))((\lambda z'. s(s(sz')))((\lambda z'. s(s(sz')))z)) \rightarrow^{\beta}$$
$$\lambda s z. (\lambda z'. s(s(sz')))((\lambda z'. s(s(sz'))) \ (s(s(sz))) \rightarrow^{\beta}$$
$$\lambda s z. (\lambda z'. s(s(sz'))) \ (s(s(s(s(s(sz)))))) \rightarrow^{\beta}$$
$$\lambda s z. s(s(s(s(s(s(s(s(sz)))))))) = 9$$

Скажем, что при $p$ и $k$ выполняется "$power \ p \ k$". Тогда докажем, что $power \ p \ (succ \ k)$ также верно.

$$power \ p \ (succ \ k) = (\lambda m n s z. \ n \ m \ s \ z) \ p \ (succ \ k) \rightarrow^{\beta}$$
$$(\lambda n s z. \ n \ p \ s \ z) \ (succ \ k) \rightarrow^{\beta}$$
$$\lambda s z. \ (succ \ k) \ p \ s \ z = \lambda s z. \ (\lambda ns'z'. \ s'(\ n \ s' \ z')) k \ p \ s \ z  \rightarrow^{\beta}$$
$$\lambda s z. \ (\lambda s'z'. \ s'(\ (\lambda s''z''. s''(s''(\dots (s''z'') \dots))) \ s' \ z')) \ p \ s \ z  \rightarrow^{\beta}$$
$$\lambda s z. \ (\lambda s'z'. \ s'(\ s'(s'(\dots (s'z') \dots)))) \ p \ s \ z  \rightarrow^{\beta}$$
$$\lambda s z. \ (\lambda z'. \ p(\ p( \ p(\dots (p \ z') \dots)))) \ s \ z  \rightarrow^{\beta}$$
$$\lambda s z. \ p(\ p( \ p(\dots (p \ s) \dots))) \ z  =$$
$$\lambda s z. \ p(\ p( \ p(\dots (\ p(p \ s)) \dots))) \ z  \rightarrow^{\beta}$$

На данный момент наше $p$ повторяется ровно $k + 1$ раз, снова мы можем провести аналогию с вложенными суммами в $mult$, но теперь в $power$ используем вложенный $mult$.

$$\lambda sz. p \ (p \ s) \ z = mult \ p \ p$$
$$\lambda s z. \ p(\ p( \ p(\dots (\ p(p \ s)) \dots))) \ z  = mult \ p \ (mult \ p (\dots (mult \ p \ p) \dots)) = p^{k+1}$$

Или же по-обычному будем подставлять значения в лямбду-абстракцию.
$$\lambda s z. \ (\lambda s'z'. s'(\dots (s'z') \dots)) \ (\ p( \ p(\dots (\ p(p \ s)) \dots))) \ z \rightarrow^{\beta}$$
$$\lambda s z. \ (\lambda z'. p( \ p(\dots (\ p(p \ s)) \dots))(\dots (\ p( \ p(\dots (\ p(p \ s)) \dots))z') \dots)) \ z \rightarrow^{\beta}$$
$$\lambda s z. \ p( \ p(\dots (\ p(p \ s)) \dots)) \ (\dots (\ p( \ p(\dots (\ p(p \ s)) \dots))z) \dots)$$

В $k+1$ позицию $s$ подставляем $p$, которое в свою очередь будет $p$-раз будет подставляться в во внешнюю значение $p$, что и является возведением в степень: $p^{k+1}$.
\end{document}
